\documentclass{article}
\usepackage[utf8]{inputenc}
\usepackage[a4paper, top=3cm, bottom=3cm, left=3cm, right=3cm]{geometry}
\usepackage{amsmath}
\usepackage{amsthm}
\usepackage{amssymb}
\usepackage{fancyhdr}
\usepackage{hyperref}
\usepackage{graphicx}
\usepackage{subfig}
\usepackage[T1]{fontenc}    % https://tex.stackexchange.com/questions/453540/
\usepackage{csquotes}       % https://www.overleaf.com/learn/latex/Typesetting_quotations
\usepackage{listings}       % https://www.overleaf.com/learn/latex/Code_listing
\usepackage{xcolor}
\usepackage{mathtools}

\definecolor{codegreen}{rgb}{0,0.6,0}
\definecolor{codegray}{rgb}{0.5,0.5,0.5}
\definecolor{codepurple}{rgb}{0.58,0,0.82}
\definecolor{backcolour}{rgb}{0.95,0.95,0.92}

\lstdefinestyle{mystyle}{
    backgroundcolor=\color{backcolour},   
    commentstyle=\color{codegreen},
    keywordstyle=\color{magenta},
    numberstyle=\tiny\color{codegray},
    stringstyle=\color{codepurple},
    basicstyle=\ttfamily\footnotesize,
    breakatwhitespace=false,         
    breaklines=true,                 
    captionpos=b,                    
    keepspaces=true,                 
    showspaces=false,                
    showstringspaces=false,
    showtabs=false,                  
    tabsize=2
}

\lstset{style=mystyle}

\pagestyle{fancy}
\fancyhf{}
\lhead{Competitive Programming}
\rhead{Week-6}
\cfoot{\thepage}

\newcommand{\B}[1]{\textbf{#1}}
\newcommand{\I}[1]{\textit{#1}}
\newcommand{\T}[1]{\texttt{#1}}
\newcommand{\Mod}[1]{\ (\mathrm{mod}\ #1)}
\newcommand{\image}[2]{\
    \begin{center}\
        \includegraphics[width = #1\linewidth]{#2}\
    \end{center}\
}

\title{\textbf{SoC 2023: Competitive Programming \\ {\Large Week-6: Range Queries}}}
\author{Mentor: Virendra Kabra}
\date{Summer 2023}

\begin{document}
\begin{sloppypar}       % for overfull, etc.

    \maketitle
    \tableofcontents
    \thispagestyle{empty}

    \newpage
    % \setcounter{page}{1}

    \section{Range Queries}
    Given an array \T{a} of size $n$, answer multiple queries of the form $(l,r)$. Common examples are sum and minimum of \T{a[l\dots r]}. Some problems may also have updates to the array.

    \section{Prefix arrays}

    For an operation $f$, $p[i] = f(a[0,\dots,i])$. Example: sum

    \begin{center}
        \begin{tabular}{|c|c|c|c|c|c|}
            \hline
            \T{a} & 1 & 3 & 0 & -2 & 5 \\
            \hline
            \T{p} & 1 & 4 & 4 & 2 & 7 \\
            \hline
        \end{tabular}
    \end{center}

    \noindent Can be used for
    \begin{itemize}
        \item Any operation with $(0,r)$ queries. Examples: max, min of ${a[0,\dots,r]}$.
        \item Invertible operations with $(l,r)$ queries. Examples: $\sum_{i=l}^{r}a[i] = \sum_{i=0}^{r}a[i] - \sum_{i=0}^{l-1}a[i]$, product.
    \end{itemize}

    \noindent Above queries are $O(1)$. Updates are costly ($O(n)$). Good with immutable arrays.\\
    Code.

    \section{Sparse Table}
    \href{https://cp-algorithms.com/data_structures/sparse-table.html}{Reference}.

    \image{0.9}{../images/st.png}

    \begin{itemize}
        \item Any number can be uniquely expressed as sum of distinct powers of two. This follows from binary representation of the number. For example, $22 = (10110)_2 = 2^4 + 2^2 + 2^1$.
        \item Similarly, any interval $[l,r]$ can be expressed as union of intervals with lengths being distinct powers of two. For example, $[5,26] = [5,20] \cup [21,25] \cup [25,26]$.
        \item Idea of sparse tables is to precompute all range queries with length being powers of two.
        \item A 2-D array $st$ is used, with $st[i][j]$ holding the result for $[j, j+2^i-1]$ (length $2^i$).
        \begin{itemize}
            \item For an array with $n$ elements, $2^i-1 < j+2^i-1 < n$, so the first dimension is $\lfloor \log_2{n}\rfloor$
            \item $[j,j+2^i-1] = [j,j+2^{i-1}-1]\cup [j+2^{i-1},j+2^i-1]$. Recurrence:
            $$st[i][j] = f(st[i-1][j], st[i-1][j+2^{i-1}])$$
        \end{itemize}
    \end{itemize}

    \noindent Above queries are $O(\log n)$. Updates would require recomputation of $st$. Again, good with immutable arrays. Works with non-invertible functions.\\
    Code.

    \section{Fenwick Tree}
    Aka Binary Indexed Tree (BIT). References - \href{https://cp-algorithms.com/data_structures/fenwick.html}{cp-algos}, \href{https://www.geeksforgeeks.org/binary-indexed-tree-or-fenwick-tree-2/}{gfg}.

    \begin{itemize}
        \item We saw that precomputing results of certain intervals helps in answering queries faster.
        \item Suppose these are computed into an array $BIT$. Let the interval be $[g(i),i]$. Then, with sum as an example, $BIT[i] = \sum_{j=g(i)}^i a[j]$.
        \item $g(i)=i$ makes $BIT=a$. Costly queries.
        \item $g(i)=0$ makes $BIT=p$. Costly updates.
        \item Pseudocode for updates and queries looks like

\begin{lstlisting}[language=C++]
query(int r):
    res = 0
    while (r >= 0):     // [g(r), r] U [g(g(r)-1), g(r)-1] U ...
        res += t[r]
        r = g(r) - 1
    return res

increase(int i, int delta):
    for all j with g(j) <= i <= j:
        t[j] += delta
\end{lstlisting}

        \item As we saw, every number can be uniquely represented as a sum of distinct powers of two. Here, length of $[g(i),i]$ is the smallest power of two in this unique representation. So, $g(i) = i-(i \& -i)+1$, where $i \& -i$ gives the least significant set bit of $i$:
        \begin{align*}
            i &= (0\dots 0)10110\\
            -i &= (1\dots 1)01010\\
            i \& -i &= (0\dots 0)00010
        \end{align*}
        We use 1-indexed implementation. Example:
        $$a[1\dots 13] = a[1\dots (1101)_2] = a[13\dots 13]\cup a[9\dots 12]\cup a[1\dots 8]$$

        \item Updates: We want to increase $a[13]$ by $5$. Then, we need to add $5$ to all $BIT[i]$ that have $a[13]$ as part of their sums. For example, $[g(14),14]=[13,14]$ but $[g(15),15]=[15,15]$. So, we add $5$ to $BIT[14]$. The idea is to repeatedly add the least significant set bit:
        $$13 \xrightarrow[]{[1101,?]} 14 \xrightarrow[]{[1110,?]} 16 \dots$$
    \end{itemize}

    \image{0.6}{../images/bit.png}

    \noindent This choice of $g(i)$ allows both queries $[1,r]$ and updates in $O(\log n)$. However, for queries $[l,r]$, invertible operations are required.\\
    Code.

    \section{Segment Tree}

    Divide-and-conquer approach. References - \href{https://cp-algorithms.com/data_structures/segment_tree.html}{cp-algos}, \href{https://www.geeksforgeeks.org/segment-tree-data-structure/}{gfg}, \href{https://codeforces.com/blog/entry/18051}{CF}.

    Example with $a = [1, 3, -2, 8, -7]$
    \image{0.5}{../images/segtree.png}
    \begin{figure}[h!]
        \begin{minipage}{.45\textwidth}
            \image{1.0}{../images/segtree-queries.png}
            \captionof{figure}{Sum $a[2\dots 4]$}
        \end{minipage}
        \begin{minipage}{.45\textwidth}
            \image{1.0}{../images/segtree-updates.png}
            \captionof{figure}{Update $a[2]$}
        \end{minipage}
    \end{figure}

    \begin{itemize}
        \item Underlying data structure is an array.
        \item Queries and updates in $O(\log n)$. Larger constants than BIT.
        \item Works with non-invertible functions as well.
    \end{itemize}

    \section{Todos}
    Check sheet.

\end{sloppypar}
\end{document}