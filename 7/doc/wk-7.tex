\documentclass{article}
\usepackage[utf8]{inputenc}
\usepackage[a4paper, top=3cm, bottom=3cm, left=3cm, right=3cm]{geometry}
\usepackage{amsmath}
\usepackage{amsthm}
\usepackage{amssymb}
\usepackage{fancyhdr}
\usepackage{hyperref}
\usepackage{graphicx}
\usepackage{subfig}
\usepackage[T1]{fontenc}    % https://tex.stackexchange.com/questions/453540/
\usepackage{csquotes}       % https://www.overleaf.com/learn/latex/Typesetting_quotations
\usepackage{listings}       % https://www.overleaf.com/learn/latex/Code_listing
\usepackage{xcolor}
\usepackage{mathtools}

\definecolor{codegreen}{rgb}{0,0.6,0}
\definecolor{codegray}{rgb}{0.5,0.5,0.5}
\definecolor{codepurple}{rgb}{0.58,0,0.82}
\definecolor{backcolour}{rgb}{0.95,0.95,0.92}

\lstdefinestyle{mystyle}{
    backgroundcolor=\color{backcolour},   
    commentstyle=\color{codegreen},
    keywordstyle=\color{magenta},
    numberstyle=\tiny\color{codegray},
    stringstyle=\color{codepurple},
    basicstyle=\ttfamily\footnotesize,
    breakatwhitespace=false,         
    breaklines=true,                 
    captionpos=b,                    
    keepspaces=true,                 
    numbers=left,                    
    numbersep=5pt,                  
    showspaces=false,                
    showstringspaces=false,
    showtabs=false,                  
    tabsize=2
}

\lstset{style=mystyle}

\pagestyle{fancy}
\fancyhf{}
\lhead{Competitive Programming}
\rhead{Week-7}
\cfoot{\thepage}

\newcommand{\B}[1]{\textbf{#1}}
\newcommand{\I}[1]{\textit{#1}}
\newcommand{\T}[1]{\texttt{#1}}
\newcommand{\Mod}[1]{\ (\mathrm{mod}\ #1)}
\newcommand{\image}[2]{\
    \begin{center}\
        \includegraphics[width = #1\linewidth]{#2}\
    \end{center}\
}

\title{\textbf{SoC 2023: Competitive Programming \\ {\Large Week-7: Flows and Tries}}}
\author{Mentor: Virendra Kabra}
\date{Summer 2023}

\begin{document}
\begin{sloppypar}       % for overfull, etc.

    \maketitle
    \tableofcontents
    \thispagestyle{empty}

    \newpage
    % \setcounter{page}{1}

    \noindent Reference: \href{https://cp-algorithms.com/graph/edmonds_karp.html}{cp-algos}, CLRS

    \section{Flows}

    \subsection{Terminology}

    \begin{itemize}
        \item Network: Directed graph with a capacity function $c:E\rightarrow \mathbb{R}^+_0$. Special vertices source $s$ and sink $t$. Analogous to a network of water pipes.
        \item Flow: Function $f$ assigning flows to edges such that for all $e\in E$, $f(e) \le c(e)$. Further, for all $u\in V\backslash \{s,t\}$, incoming and outgoing flows must be equal:
        $$\sum_{(v,u)\in E}f((v,u)) = \sum_{(u,v)\in E}f((u,v))$$
        \item Can deduce $$\sum_{(s,u)\in E}f((s,u)) = \sum_{(u,t)\in E}f((u,t))$$
    \end{itemize}

    \image{0.6}{../images/Flow1.png}

    \begin{itemize}
        \item Value of a flow: Sum of outgoing flows from source. Equivalent to sum of incoming flows to sink.
        \item Maximal flow: Flow with maximum value.
    \end{itemize}

    We need to find a maximal flow.

    \subsection{Ford-Fulkerson Method}

    \begin{itemize}
        \item Residual capacity of an edge: $r(e) = c(e) - f(e)$. By an earlier constraint, this is non-negative.
        \item Residual network: Add (directed) reverse edges, with zero capacity and $-f(e)$ flow. Note that reverse edges also have non-negative residual flow. These can be used to reverse an earlier assigned flow. For example, $(A,D)$ has a residual capacity of 3. ``Using'' it is same as undoing $(D,A)$ flow:\\
        \begin{minipage}{.45\textwidth}
            \image{1.0}{../images/Flow7.png}
        \end{minipage}
        \begin{minipage}{.45\textwidth}
            \image{1.0}{../images/Flow8.png}
        \end{minipage}
        \item Augmenting path: A simple path in the residual network, along edges with positive residual capacity.
        \item Pseudocode
\begin{lstlisting}[]
while there is an augmenting path P from s to t:
    c = capacity of P
    for e in P:
        r(e) -= c
        r(e_reverse) += c
max_flow = sum of outgoing flows from s
\end{lstlisting}
        \item Edmonds-Karp implementation: Use BFS to find augmenting paths.
    \end{itemize}

    \subsection{Example: \href{https://www.geeksforgeeks.org/maximum-bipartite-matching/}{Maximum Bipartite Matching}}
    \begin{itemize}
        \item X applicants and Y jobs. Each applicant $x$ applies for some set of jobs $Y_x$. Each job can only admit one applicant, and each applicant can only get one job. Find an assignment of jobs that fills maximum number of vacancies.
        \item Matching of a graph: Set of edges such that on any vertex, at most one of these is incident.
        \item Maximum matching: Any matching with the largest number of edges.
        \item We represent this as a bipartite graph, and the problem reduces to find its maximum matching.
        \image{0.6}{../images/maximum_matching1.png}
        \begin{minipage}{.45\textwidth}
            \image{1.0}{../images/maximum_matching2.png}
        \end{minipage}
        \begin{minipage}{.45\textwidth}
            \image{1.0}{../images/maximum_matching21.png}
        \end{minipage}
    \end{itemize}

    \section{Tries}
    \noindent Reference: \href{https://www.geeksforgeeks.org/trie-insert-and-search/}{gfg}.\\
    \noindent To efficiently store and search strings.

    \image{0.6}{../images/trie1.png}

    \noindent Insertion
    \image{0.7}{../images/trie2.png}

    \noindent Search
    \image{0.7}{../images/trie3.png}

    \section{Todos}
    Check sheet.

\end{sloppypar}
\end{document}